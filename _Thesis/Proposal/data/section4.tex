%%%%%%%%%%%%%%%%%%%%%%%%%%%%%%%%%%%%%%%%%%%%%%%%%%%
%
%  New template code for TAMU Theses and Dissertations starting Fall 2016.  
%
%
%  Author: Sean Zachary Roberson
%  Version 3.17.09
%  Last Updated: 9/21/2017
%
%%%%%%%%%%%%%%%%%%%%%%%%%%%%%%%%%%%%%%%%%%%%%%%%%%%
%%%%%%%%%%%%%%%%%%%%%%%%%%%%%%%%%%%%%%%%%%%%%%%%%%%%%%%%%%%%%%%%%%%%%%
%%                           SECTION IV
%%%%%%%%%%%%%%%%%%%%%%%%%%%%%%%%%%%%%%%%%%%%%%%%%%%%%%%%%%%%%%%%%%%%%



\chapter{EXPECTED RESULTS}
The following are the expected results for each objective of this research.
\section{Surface Representation Data Types}

Multiple types of representation are expected to be found during this objective. The purpose of this objective is to identify ideal representation types that will aid in fulfilling the remaining objectives.

The results from this objective are expected to be:

\begin{itemize}
	\item One or more surface representation types that can be used to describe the data from VTMS
	\item Insight into conventions of multiple industries that use penetration and collision detection algorithms
	\item Inspiration on multiple types of identification and resolution methods
\end{itemize}

\section{Identification of Interpenetration Regions}

Properly identifying the inter-penetration regions of the surface representations is vital to completing the third object of this research. Without proper identification, there can not be a complete resolution of the inter-penetration regions.

The results from this objective are expected to be:
\begin{itemize}
	\item An algorithm that accurately detects inter-penetration regions for multiple representation types
	\item An algorithm that collects the required data to resolve inter-penetrations
	\item A methodology to accurately visualize interpenetration regions
\end{itemize}

\section{Resolution of Interpenetration Regions}

Completion of this research is synonymous with resolving inter-penetrations or determining the remaining steps for the most realistic solution. The best solution to resolving these inter-penetrations may vary depending on the situation.

The results from this objective are expected to be:
\begin{itemize}
	\item An algorithm that resolves inter-penetrations for one or more of the chosen representation types
	\item Recommendations for the usage of algorithms and methods developed during this research
	\item A method that exports useful inter-penetration data for user discretion
	\item A suite of software that can be easily implemented for other users
\end{itemize}

