%%%%%%%%%%%%%%%%%%%%%%%%%%%%%%%%%%%%%%%%%%%%%%%%%%%
%
%  New template code for TAMU Theses and Dissertations starting Fall 2016.  
%
%
%  Author: Sean Zachary Roberson
%  Version 3.17.09
%  Last Updated: 9/21/2017
%
%%%%%%%%%%%%%%%%%%%%%%%%%%%%%%%%%%%%%%%%%%%%%%%%%%%
%%%%%%%%%%%%%%%%%%%%%%%%%%%%%%%%%%%%%%%%%%%%%%%%%%%%%%%%%%%%%%%%%%%%%
%%                           ABSTRACT 
%%%%%%%%%%%%%%%%%%%%%%%%%%%%%%%%%%%%%%%%%%%%%%%%%%%%%%%%%%%%%%%%%%%%%

\chapter*{ABSTRACT}
\addcontentsline{toc}{chapter}{ABSTRACT} % Needs to be set to part, so the TOC doesnt add 'CHAPTER ' prefix in the TOC.

\pagestyle{plain} % No headers, just page numbers
\pagenumbering{roman} % Roman numerals
\setcounter{page}{2}

As the usage of computational analysis for composite design becomes increasingly more popular, the desire to create and test textile composite materials is increasing as well. Previous research and study has been conducted using idealized woven textile geometries that lack the realism of imperfect woven composite fiber bundles (or tows). For most analyses, a filament (bundles of fibers) discretization of the tow is too complex to be analyzed, so instead, a surface representation of the tows is created so that homogenized properties can be applied. This representation is an approximation of the fiberized tow. When theses surfaces are created, small regions of inter-penetrations are formed where the surfaces cross into each other. This creates two problems: a physically impossible occupation of the same space concerning the tows and incompatibility of the mesh resulting from these meshes crossing through each other.


This research proposal will outline a plan to address these problems and resolve them. The following three objectives are proposed rectify this issue: (1) Distinguish between data types that can be used to define the surface geometry, (2) Identify the regions of inter-penetrations between the surfaces, and (3) Discuss resolution to the interpenetration regions. The results of this study could result in more realistic woven textile geometries for computational testing of these complex composites that are compatible with traditional finite element analysis.


 

\pagebreak{}
